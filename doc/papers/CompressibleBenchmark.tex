\section{Compressible benchmark}
The set of non dimensional equations that have to be solved,
\begin{eqnarray}
 \nabla \cdot (\rho_{r}\bf u) = 0 \\
-\nabla P + \nabla \cdot \tau = Ra \bf e_{z} T
\end{eqnarray} 

The model setup to reproduce the compressible benchmark of Tan and Gurnis is as follows

\subsection{Model domain}
The model domain is a unit box with 16, 32 or 64 elements in each direction (equal spacing)
free slip boundary conditions are applied.
\subsection{boundary conditions}
impermeable free slip boundary conditions are applied to all boundaries.
\subsection{viscosity}
the viscosity variations are restricted to the vertical direction (1D) to be able to do a FFT decomposition of the equation allowing for a semi analytical solution.
The 1D viscosity profile is given as,
\begin{equation}
\eta = e^{az}
\end{equation} 
with $a$ a constant either 0 or 2 
\subsection{Right hand side}
The model is driven by a lateral temperature perturbation of the form 
\begin{equation}
T(x,z) = sin(\pi z)cos(\pi k x)
\end{equation}
with k the wavenumber
\subsection{Density}
the reference  density is expressed as 
\begin{equation}
\rho_{r}(z) = e^{\beta(1-z)}
\end{equation} 
with $\beta = Di/\gamma$ \\
for the  incompressible Bousinesq approximation  (BA) $Di = 0$ and $\gamma = \inf$ for the truncated anelastic approximation (TALA) $Di = 0.5$ and $\gamma = 1$.
The density anomaly is given by,
\begin{equation}
\Delta \rho (x,z) = \rho_{r}(z)T(x,z)
T(x,z) = sin(\pi z)cos(\pi k x)
\end{equation}
The analytical solution to the flow is given as
\begin{equation}
\partial \left[ 
\begin{array}{c}
U_{z} \\
U_{x} \\
\sum_{zz}/2\eta_{0}k  \\
\sum_{xz}/2\eta_{0}k  \\            
\end{array}
\right ]
=
\left[ 
\begin{array}{cccc}
 \beta & -k & 0 & 0 \\
k &  0 & 0 & 2 k / \eta^{*} \\
0 & 0 & 0 & -\kappa \\
-\beta \eta^{*} & 2\kappa \eta^{*} & k & 0 \\
\end{array}
\right ]
\cdot
\left[ 
\begin{array}{c}
U_{z} \\
U_{x} \\
\sum_{zz}/2\eta_{0}k  \\
\sum_{xz}/2\eta_{0}k  \\   
\end{array}
  \right ]
+
\left[ 
\begin{array}{c}
0 \\
0 \\
\Omega Ra / 2 \eta_{0} k  \\
0  \\   
\end{array}
  \right ]
\end{equation}
 